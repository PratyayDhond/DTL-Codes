
\documentclass[12pt]{article}
\author{pratyayDhond}
\title{LaTeX}
\begin{document}

\maketitle

\tableofcontents 

\section{Latex and it's uses}
\vspace{.2in}

\paragraph{ \textit{LaTeX} is one of the best, most prominent and an open source software developed for the purposed of Making the task of writing and making documents easier }

\subsection{What is LaTeX?}
		\textbf{•} LaTeX is a document preparation system used for the communication and publication of scientific documents. \\
		\textbf{•} The LaTeX software is one of the best available tools for Scientific reports writing and other media. 
		\paragraph { \textit{LaTeX} is a tool for creating professional-looking documents. Originally based off the WYSIWYM (What You See Is What You Mean) idea, it only focuses on the contents of your document, while the computer will take care of the rest. Instead of spacing out text on a page, or any other formating, used in Microsoft Word, LibreOffice Writer etc.*, users can enter plain text, letting LaTeX do all the work. }

\newpage
\subsection{Why use Latex?}

	\paragraph{LaTex isn't only used for resumes, but all over the world for scientific documents, books, and other various forms of publishing. It simultaneously creates organized typeset documents and allows users to focus on the more complicated parts of typesetting, which can included but is not limited to, inputting mathematics, creating tables of contents, referencing and creating bibliographies, and having a consistent layout across all sections. There are no fine lines when utilizing with Latex due to the huge number of open source packages available. These packages allow users to do more with LATEX, like adding footnotes, draw schematics, create tables, and more.}
	
	\paragraph{One reason why people use LaTeX is that it separates the document content from style. This means that we can change the documents appearance with ease after you have the written content. Additionally, you can create one document style which can be used similar to a template for other documents. This allows scientific journals to create templates for submissions. There are hundreds, if not thousands, of templates available online for everything!}

\subsection{What are the uses of LaTeX?}
	\begin{itemize}
	\item{Journal Writing} 
	\item{Typesetting of \textbf{complex mathematical formulas}}
	\item{Inclusion of \textbf{Artwork} and images in your latex file}	
	\end{itemize}

\end{document}
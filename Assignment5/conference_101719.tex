\documentclass[conference]{IEEEtran}
\IEEEoverridecommandlockouts
% The preceding line is only needed to identify funding in the first footnote. If that is unneeded, please comment it out.
\usepackage{cite}
\usepackage{amsmath,amssymb,amsfonts}
\usepackage{algorithmic}
\usepackage{graphicx}
\usepackage{textcomp}
\usepackage{xcolor}
\def\BibTeX{{\rm B\kern-.05em{\sc i\kern-.025em b}\kern-.08em
    T\kern-.1667em\lower.7ex\hbox{E}\kern-.125emX}}
\begin{document}

\title{GIT and GitHub\\
\thanks{Identify applicable funding agency here. If none, delete this.}
}

\author{\IEEEauthorblockN{1\textsuperscript{st} Pratyay Dhond}
\IEEEauthorblockA{\textit{Computer Engineering} \\
\textit{COEP}\\
Pune, India\\
dhondpratyay@gmail.com}
\and
\IEEEauthorblockN{2\textsuperscript{nd} Priyanshu Lapkale}
\IEEEauthorblockA{\textit{Computer Engineering} \\
\textit{VIIT}\\
Pune, India\\
lapkale0110@gmail.com}
\and
\IEEEauthorblockN{3\textsuperscript{rd} Manthan Ghonge}
\IEEEauthorblockA{\textit{Information Technology} \\
\textit{VIIT}\\
Pune, India\\
manthanghonge123@gmail.com}
}

\maketitle

\begin{abstract}
Git is a free and open-source distributed version control system designed to handle everything from small to very large projects with speed and efficiency. It is a tool that helps developers keep track of the changes they make to the code over time, allowing them to collaborate with other developers, revert to previous versions of the code, and easily identify the source of bugs.
\end{abstract}

\section{GIT Basics}

\subsection{How does git work?}

Git works by keeping a local copy of the entire codebase on each developer's computer. When a developer makes changes to the code, they can use Git to "commit" those changes to their local copy, creating a new version of the code. They can then "push" those changes to a remote repository, which is a centralized location where all developers can access the latest version of the code. Other developers can then "pull" those changes down to their own local copies, and continue working on the code.

\section{What is GIT and GitHub}
Git is a distributed version control system for tracking changes in source code during software development. It was created by Linus Torvalds in 2005. Git allows multiple developers to work on the same codebase simultaneously, and it keeps track of every version of the code, making it easy to roll back to previous versions or to see who made specific changes. Git is widely used in the software development community, and it is the backbone of many development collaborations and open-source projects. It's also becoming a standard tool for any kind of text-based content management.

\subsection{What is GitHub}
GitHub is a web-based platform that is built on top of Git. It provides a central location for developers to store, manage, and collaborate on Git repositories. Some of the key features of GitHub include:

- Remote storage: Developers can use GitHub to store their code and collaborate with other developers, regardless of their physical location.

- Web-based interface: GitHub provides a web-based interface that makes it easy to view and manage code, track issues, and collaborate with other developers.

- Issues and pull requests: GitHub provides tools for developers to track and manage issues and pull requests, which are requests for changes to the code.

- Branching and merging: GitHub provides powerful branching and merging tools that make it easy for developers to work on multiple features or bug fixes at the same time, without interfering with each other's work.

- Collaboration: GitHub makes it easy for developers to collaborate on code, with features like code reviews, team management, and access control.

- Third-party integrations: GitHub has a wide range of third-party integrations, such as continuous integration tools, that can be used to automate the development process.

- Open-source hosting: GitHub is widely used as a hosting platform for open-source software. With GitHub, it's easy to make your code available to the public and receive contributions from other developers.

\subsection{GIT Commands}
The various GIT Commands are: 
git init, git clone, git add, git commit, git status, git diff, git log, git branch, git checkout, git merge, git pull, git push.
\begin{itemize}
\item git init: Initializes a new Git repository.
\item git clone: Copies an existing Git repository from a remote location to your local machine.  
\item git add: Adds one or more files to the "staging area" in preparation for a commit. 
\item git commit: Creates a new "commit" in the repository, which is a snapshot of the current state of the code. 
\item git status: Shows the current status of the repository, including which files have been modified and which files are in the staging area.
\item git diff: Shows the differences between the current state of the code and the last commit.
\item git log: Shows a log of all commits made to the repository, along with the commit messages.
\item git branch: Lists all branches in the repository and highlights the current branch.
\item git checkout: Switches to a different branch, or a specific commit.
\item git merge: Merges changes from one branch into another.
\item git pull: Fetches the latest changes from a remote repository and merges them into the local branch. 
\item git push: Pushes the local commits to a remote repository.
\end{itemize}

\end{document}
